\documentclass{article}

\usepackage[scale = 0.9]{geometry}
\usepackage{hyperref}

\begin{document}
\section{\href{https://stackoverflow.com/questions/41573587/what-is-the-difference-between-venv-pyvenv-pyenv-virtualenv-virtualenvwrappe}{Python virtual environment}}
\begin{itemize}
\item standard libraries: venv
\item third party libraries: virtualenv, pyenv, pyenv-virtualenv, virtualenvwrapper...\\
this project uses \textbf{pipenv}, because 
	\begin{enumerate}
		\item this is the recommended tool for application dependency management by python.org. It combines Pipfile, pip and virtualenv into one command
		\item It replaces the requirements.txt and especially solves some issues
that use requirements.txt in multiple environments such as test integration and production. This means we are able to manage virtual environments and packages only with one tool.
	\end{enumerate}
\item \href{https://levelup.gitconnected.com/beginners-guide-to-pipenv-9340a6c35147}{commands related to use pipenv}
	\begin{itemize}
		\item install pipenv: \\\verb|python3 -m pipenv shell --python /Library/Frameworks/Python.framework/Versions/3.11/bin/python3|
		\item check the location of virtual environment: \verb|pipenv --venv|
		\item install package: \verb|pipenv install pandas|
		\item run \verb|jupyter notebook| inside pipenv environment:
		\verb|pipenv run jupyter notebook|
	\end{itemize}
\item \href{https://geekflare.com/aws-s3-command-examples/}{AWS setup}
	\begin{itemize}
		\item install aws cli: \verb|pipenv install awscli|
	\end{itemize}
\item \href{https://towardsdatascience.com/introduction-to-pythons-boto3-c5ac2a86bb63}{connect aws s3 using \texttt{boto3}}
	\begin{itemize}
		\item 
	\end{itemize}
\item \href{https://pbpython.com/groupby-agg.html}{pandas aggregation command}
	\begin{itemize}
		\item there are multiple ways to call an aggregation function, like list, dictionary and tuple
		\item The major distinction to keep in mind is that \texttt{count} will not include \texttt{NaN} values whereas \texttt{size} will. 
	\begin{verbatim}
	agg_func_count = {'embark_town': ['count', 'nunique', 'size']}
df.groupby(['deck']).agg(agg_func_count)
	\end{verbatim}
	\end{itemize}


\end{itemize}
\end{document}